\documentclass[final]{ieee}

\usepackage{microtype} %This gives MUCH better PDF results!
\usepackage[cmex10]{amsmath}
\usepackage{amssymb}
\usepackage{fnbreak} %warn for split footnotes
\usepackage{url}
\usepackage[utf8]{inputenc} % allows to write Faugere correctly
\usepackage[bookmarks=true, citecolor=black, linkcolor=black, colorlinks=true]{hyperref}
\hypersetup{
pdfauthor = {Mate Soos},
pdftitle = {CryptoMiniSat + CCAnr},
pdfsubject = {SAT Competition 2020},
pdfkeywords = {SAT Solver, DPLL, SLS}}
%\usepackage{butterma}

%\usepackage{pstricks}
\usepackage{graphicx,epsfig,xcolor}

\begin{document}
\title{Combined SLS and CDCL instances at the SAT Competition 2020}
\author{Mate Soos (National University of Singapore)}

\maketitle
\thispagestyle{empty}
\pagestyle{empty}



\section{Overview}
System generates problems that are a combination of an easy satisfiable random problem and an easily satisfiable structured problem. It combines these two satisfiable problems in a way to make sure their solutions overlap on OVERLAP number of variables, where OVERLAP is at least 1. This makes sure the problem cannot easily be cut into 2 pieces, but the final CNF is satisfiable.

The system tries to mimic real-world problems that are combinations of two easy problems, easily solvable separately by two separate types of systems. However, the combination of them, truly combined but not highly overlapping is hard to solve.


\section{Components}


\subsection{Fixed Clause Length Model Random CNF Generator}

As part of the well-known WalksSAT~\cite{Selman95localsearch} package, \texttt{makewff} generates random formulas using the fixed clause length model. See WalkSAT and its source code for details. Used parameters: \texttt{"makewff -seed SEED -cnf 3 NVARS (4.25*NVARS)"} where \texttt{NVARS} was varied between 300 and 500, and competition-matching values were picked out.

\subsection{Grain of Salt}
The system uses Grain of Salt~\cite{soos2010grain} to generate random problem from the CRYPTO-1 stream cipher, with preset number of outputs, a random IV and random key. When the number of \texttt{OUTPUTS} is low, it always generates a satisfiable problem. Increasing the number of \texttt{OUTPUTS} creates progressively harder problems which eventually become unsatisfiable. Used parameters: \texttt{"grainofsalt --seed SEED --crypto crypto1 --outputs OUTPUTS --karnaugh 0 --init no --num 1"} where \texttt{OUTPUTS} was set between 30 and 50, and competition-matching values were picked out.


\subsection{Multipart Combiner with Overlap}
This is tool takes two satisfiable problems and their solutions, plus the parameter \texttt{OVERLAP}. It outputs a solution that is a combination of the two CNF's clauses, but the variables have been manipulated to be separate for both CNFs except for \texttt{OVERLAP} number of variables. These \texttt{OVERLAP} variables are specially picked from the solutions: the solutions provided must have the same setting for these variables. This ensures that the output is a satisfiable CNF and that the 2 problems cannot easily be cut apart. Used parameters: \texttt{"multipart-match-sol.py CNF1 CNF2 CNF1-solution CNF2-solution OVERLAP"}


\section{Historical background}
Early attempts at hybrid solving, such as ReasonLS~\cite{shaoweixindi} combined SLS and CDCL solvers in shell-scripted, non-cohesive way. However, it ignited a very interesting development, culminating in CaDiCaL~\cite{biere2018cadical} and CryptoMiniSat~\cite{CMS} both having a hybrid SLS-CDCL strategy in the SAT Race 2019. This combined hybrid strategy has been proven useful in industrial instances. The CNFs generated by our system effectively tests whether a hybrid strategy is employed by the underlying solver.

\section{Rationale}
These CNFs were mostly created to encourage hybrid solvers such as SLS+CDCL, Gauss-Jordan elimination+CDCL, Groebner Basis+CDCL, etc. Note that with linearization, Gauss-Jordan elimination is capable of solving non-linear problems as well, and is therefore not restricted to XOR constraints. These hybrid systems could potentially prove very useful for the SAT community in the long run.

\bibliographystyle{splncs03}
\bibliography{sigproc}

\vfill
\pagebreak

\end{document}
